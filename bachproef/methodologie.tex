%%=============================================================================
%% Methodologie
%%=============================================================================

\chapter{\IfLanguageName{dutch}{Methodologie}{Methodology}}%
\label{ch:methodologie}

%% TODO: In dit hoofstuk geef je een korte toelichting over hoe je te werk bent
%% gegaan. Verdeel je onderzoek in grote fasen, en licht in elke fase toe wat
%% de doelstelling was, welke deliverables daar uit gekomen zijn, en welke
%% onderzoeksmethoden je daarbij toegepast hebt. Verantwoord waarom je
%% op deze manier te werk gegaan bent.
%% 
%% Voorbeelden van zulke fasen zijn: literatuurstudie, opstellen van een
%% requirements-analyse, opstellen long-list (bij vergelijkende studie),
%% selectie van geschikte tools (bij vergelijkende studie, "short-list"),
%% opzetten testopstelling/PoC, uitvoeren testen en verzamelen
%% van resultaten, analyse van resultaten, ...
%%
%% !!!!! LET OP !!!!!
%%
%% Het is uitdrukkelijk NIET de bedoeling dat je het grootste deel van de corpus
%% van je bachelorproef in dit hoofstuk verwerkt! Dit hoofdstuk is eerder een
%% kort overzicht van je plan van aanpak.
%%
%% Maak voor elke fase (behalve het literatuuronderzoek) een NIEUW HOOFDSTUK aan
%% en geef het een gepaste titel.

Het onderzoek naar centrale truststorebeheer zal opgedeeld worden in drie fases: een literatuurstudie, praktijkstudie en uiteindelijke rapportage van de configuraties die gebruikt werden. \\


In de eerste anderhalve maand zal de eerste fase van het onderzoek uitgevoerd worden, de literatuurstudie.

Het doel in deze fase is om een diepgaand begrip te krijgen over trust management, certificaatbeheer en de verschillende aanpakken bij truststorebeheer.

In deze literatuurstudie zal er gekeken worden naar de bestaande theorieën, technologieën en andere tools die in de realiteit worden gebruikt.

Ook zal er gekeken worden naar de uitdagingen van trust management in een omgeving met netwerksegmentatie en diverse besturingssystemen.

Doorheen deze fase zal er periodiek overleg zijn met de co-promotor om de voortgang te bespreken en een mogelijkse oplossing te kiezen die verder kan worden uitgewerkt in de praktijkstudie.\\


De info die verkregen wordt in deze literatuurstudie zal de basis leggen voor de volgende fase van dit onderzoek: de praktijkstudie.

Deze fase zou een totale tijdsduur van 6 weken hebben. Deze praktijkstudie opgedeeld in 3 delen.

het ontwerpen van een virtuele omgeving, de implementatie van het centraal truststorebeheer en de evaluatie van de oplossing. \\


In de eerste stap zal er weer worden gekeken naar veel gebruikte software en besturingssystemen binnen bedrijven om zo een realistische infrastructuur te creëren.

Ook wordt er een certificate authority (CA) opgezet om het beheer van certificaten en een interne private PKI te simuleren.

Deze virtuele omgeving zal worden opgezet binnen GNS3, een netwerkemulator die het mogelijk maakt om complexe netwerken te simuleren. 

De volledige opzet van het netwerk zal een week duren. \\


De tweede stap, de implementatie van een centraal trust management systeem, begint eind maart en zal vijf weken duren.

In deze fase wordt een gecentraliseerde oplossing voor trust management opgezet. Dit gebeurt door middel van de tools die gekozen werden na de literatuurstudie in combinatie met eventuele zelfgeschreven scripts voor het certificaatbeheer.

Het doel hier is dat bij het toevoegen of verwijderen van een nieuw root certificaat, de truststores op elk systeem in het netwerk die binnen deze laag trust valt dat root certificaat bevat. \\


Tijdens de implementatie vindt ook de derde stap van de praktijkstudie plaats, die bestaat uit de evaluatie en validatie van de oplossing.

Deze fase begint na het succesvol implementeren van een oplossing.

De centrale vraag is hoe kwaliteitsvol de oplossing is en of deze voldoet aan de verwachtingen en ook om te kijken wat de oplossing niet heeft kunnen bereiken.

Dit wordt bepaald door testcases op te stellen om de functionaliteiten te testen, zoals verwijderen of toevoegen van root certificaten binnen bepaalde netwerksegmenten.

Ook wordt de schaalbaarheid van de oplossing in vraag gesteld.

Bij de schaalbaarheid zal er gekeken worden naar hoe de oplossing omgaat met het veranderen of uitbreiden van het netwerk alsook de groei van het aantal root certificaten.

Hierna wordt ook nagedacht over hoe de tekortkomingen van de oplossing kunnen worden opgelost of welke alternatieven er zijn om deze tekortkomingen te vermijden. \\


Als laatste fase van dit onderzoek zal er een rapportage gemaakt worden binnen deze paper onder het hoofdstuk `Proof-of-concept'. Dit zal een tijdsduur van 4 weken hebben. De rapportage zal de configuraties van de oplossingen binnen de PoC beschrijven alsook de positieve en negatieve punten van deze oplossing.

Op basis van de bevindingen worden dan conclusies en aanbevelingen geformuleerd voor de implementatie van centraal truststorebeheer alsook welke aspecten mogelijks verder onderzoek vereisen.


