%%=============================================================================
%% Methodologie
%%=============================================================================

\chapter{\IfLanguageName{dutch}{Methodologie}{Methodology}}%
\label{ch:methodologie}

%% TODO: In dit hoofstuk geef je een korte toelichting over hoe je te werk bent
%% gegaan. Verdeel je onderzoek in grote fasen, en licht in elke fase toe wat
%% de doelstelling was, welke deliverables daar uit gekomen zijn, en welke
%% onderzoeksmethoden je daarbij toegepast hebt. Verantwoord waarom je
%% op deze manier te werk gegaan bent.
%% 
%% Voorbeelden van zulke fasen zijn: literatuurstudie, opstellen van een
%% requirements-analyse, opstellen long-list (bij vergelijkende studie),
%% selectie van geschikte tools (bij vergelijkende studie, "short-list"),
%% opzetten testopstelling/PoC, uitvoeren testen en verzamelen
%% van resultaten, analyse van resultaten, ...
%%
%% !!!!! LET OP !!!!!
%%
%% Het is uitdrukkelijk NIET de bedoeling dat je het grootste deel van de corpus
%% van je bachelorproef in dit hoofstuk verwerkt! Dit hoofdstuk is eerder een
%% kort overzicht van je plan van aanpak.
%%
%% Maak voor elke fase (behalve het literatuuronderzoek) een NIEUW HOOFDSTUK aan
%% en geef het een gepaste titel.

Het onderzoek naar centrale truststorebeheer zal opgedeeld worden in drie fases: een literatuurstudie, praktijkstudie en uiteindelijke rapportage en oplevering.

In de eerste anderhalve maand zal de eerste fase van het onderzoek uitgevoerd worden, de literatuurstudie.

Het doel in deze fase is om een diepgaand begrip te krijgen over trust management, certificaatbeheer en de verschillende aanpakken bij truststorebeheer.

In deze literatuurstudie zal er gekeken worden naar de bestaande theorieën, technologieën en andere tools die momenteel worden gebruikt in praktijk.

Ook zal er gekeken worden naar de uitdagingen van trust management in een omgeving met netwerksegmentatie en diverse applicaties en besturingssystemen.

Doorheen deze fase zal er periodiek overleg zijn met de co-promotor om de voortgang te bespreken en een mogelijkse oplossing te kiezen die verder kan worden uitgewerkt in de praktijkstudie.

De info die verkregen wordt in deze literatuurstudie zal de basis leggen voor de volgende fase van dit onderzoek: de praktijkstudie.

Deze fase zou starten halverwege maart en loopt tot eind april met een duur van 6 weken. Deze praktijkstudie opgedeeld in 3 delen.

het ontwerpen van een virtuele omgeving, de implementatie van het centraal truststorebeheer en de evaluatie van de oplossing.

In de eerste stap zal er weer worden gekeken naar veel gebruikte software en besturingssystemen binnen bedrijven om zo een realistische infrastructuur te creëren.

Ook wordt er een certificate authority (CA) opgezet om het beheer van certificaten en een interne private PKI te simuleren.

Deze virtuele omgeving zal worden opgezet binnen GNS3, een netwerkemulator die het mogelijk maakt om complexe netwerken te simuleren.

De tweede stap, de implementatie van centraal truststorebeheer, begint eind maart en duurt drie weken.

In deze fase wordt een gecentraliseerde oplossing voor truststorebeheer opgezet. Dit gebeurt door middel van de tools die gekozen werden samen met de co-promotor in de literatuurstudie in combinatie met eventuele zelfgeschreven scripts voor het certificaatbeheer.

Het doel hier is dat bij het genereren of updaten van een nieuw root certificaat, de truststores op elk systeem in het netwerk en zijn segmenten consistent zijn. Ook zullen er veiligheidscontroles geïmplementeerd worden, zoals het monitoren voor ongeldige of verlopen certificaten.

Na de implementatie volgt de derde stap van de praktijkstudie, die bestaat uit de evaluatie en validatie van de oplossing.

Deze fase begint begin halfweg april en duurt 2 weken.

De centrale vraag is hoe kwaliteitsvol de oplossing is en of deze voldoet aan de verwachtingen.

Er worden testcases opgesteld om veelvoorkomende uitdagingen te simuleren, zoals het herroepen van een rootcertificaat of het omgaan met verlopen certificaten.

Ook wordt de schaalbaarheid en efficiëntie van de oplossing getest.

Bij de schaalbaarheid zal er gekeken worden naar hoe de oplossing omgaat met het veranderen of uitbreiden van het netwerk. Bij de efficiëntie zal er gekeken worden naar hoe snel aanpassingen van certificaten binnen de centrale truststore verspreid worden binnen het netwerk zonder het verstoren van andere diensten.

Als laatste fase van dit onderzoek zal er een rapportage gemaakt worden. Dit zal gebeuren in het begin van mei met een tijdsduur van 4 weken. De rapportage zal de ontwerpkeuzes en oplossing binenn de PoC beschrijven alsook de evaluatie en resultaten van deze oplossing.

Op basis van de bevindingen worden dan conclusies en aanbevelingen geformuleerd voor de implementatie van centraal truststorebeheer alsook welke aspecten mogelijks verder onderzoek vereisen.


