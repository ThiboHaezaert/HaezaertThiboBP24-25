%%=============================================================================
%% Voorwoord
%%=============================================================================

\chapter*{\IfLanguageName{dutch}{Woord vooraf}{Preface}}%
\label{ch:voorwoord}

%% TODO:
%% Het voorwoord is het enige deel van de bachelorproef waar je vanuit je
%% eigen standpunt (``ik-vorm'') mag schrijven. Je kan hier bv. motiveren
%% waarom jij het onderwerp wil bespreken.
%% Vergeet ook niet te bedanken wie je geholpen/gesteund/... heeft

In de voorbije 3 jaren van mijn studies in toegepaste informatica heb ik veel mogen bijleren, dankzij deze kennis mocht mijn passie voor IT nog meer groeien. 
In mijn laatste jaar kreeg ik de kans om bij KBC mijn stage te lopen, waar ik al gauw zag dat naast mijn opleiding nog veel kon worden bijgeleerd. \\

Binnen deze stage leerde ik veeltal bij over certificate authorities en public key infrastructures, technologieën die een cruciale rol hebben in de werking van digitale communicatie, maar waarvan ik alleen beschikte over de basiskennis.
Mijn co-promotor, Dirk Mussen, wie ook mijn stagementor was, bracht mij dan ook het idee voor het onderzoeksonderwerp van deze bachelorproef. Een onderwerp dat een actueel probleem aanhaalt die vele bedrijven en organisaties treft.
Dankzij de kennis opgedaan tijden mijn opleiding, stage en de begeleiding van mijn co-promotor, was het begrijpen en aanpakken van dit probleem een stuk eenvoudiger. \\

Bij deze wil ik mijn co-promotor graag bedanken voor zijn begeleiding en steun tijdens het uitvoeren van dit onderzoek. 
Naast mij co-promotor bedank ik ook graag mijn promotor, Gilles Blondeel voor het geven van feedback op deze paper alsook tips voor het mogelijks presenteren van deze bachelorproef.
Ik bedankt ook graag alle andere docenten die ik doorheen mijn opleiding heb mogen ontmoeten, voor de kennis die zij met mij hebben gedeeld.