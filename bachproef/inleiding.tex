%%=============================================================================
%% Inleiding
%%=============================================================================

\chapter{\IfLanguageName{dutch}{Inleiding}{Introduction}}%
\label{ch:inleiding}

In moderne bedrijfsomgevingen worden netwerken steeds complexer en diverser. Organisaties bevatten vaak heterogene netwerken, dat wilt zeggen dat binnen het netwerk verschillende soorten apparaten aanwezig zijn die niet allemaal dezelfde besturingssystemen en software gebruiken. 
Vaak zijn deze netwerken ook opgedeeld in verschillende segmenten op basis van de bedrijfsrollen en -vereisten, om zo de aanvalshoeken te beperken en de veiligheid van het netwerk te verhogen.
Een andere preventieve maatregelen die bedrijven vaak nemen is het limiteren van de root certificaten die vertrouwd worden door een systeem, zodanig het enkel de root certificaten vertrouwd die cruciaal zijn voor de werking van het systeem. \\

Een specifiek probleem binnen deze context is dan hoe men de inhoud van de verschillende truststores over verschillende besturingssystemen kan beheren op een manier waarbij elk systeem enkel de root certificaten vertrouwt die noodzakelijk zijn voor dat systeem zelf. 
Trust managment wilt dit probleem oplossen door te beheren wie welke root certificaten vertrouwt (welke trust een systeem heeft) en hoe dit kan worden afgedwongen.

\section{\IfLanguageName{dutch}{Probleemstelling}{Problem Statement}}%
\label{sec:probleemstelling}

Vele bedrijven met grotere diverse netwerken zoals heterogene netwerken en gesegmenteerde netwerken hebben vandaag de dag nogsteeds moeite met het centraal beheren van de (root) certificaten die worden vertrouwd door hun end-points. 
Dit komt door de verspreide trust stores die gevormt worden door de verschillende gebruikte applicaties en besturingssystemen binnen het netwerk, daarnaast is het ook belangrijk om het aantal vertrouwde certificaten te beperken om de veiligheid van het netwerk te garanderen.
Bedrijven hebben dan ook netwerksegmentatie die bepaald is op basis van de rol van de systemen binnen het netwerk, maar ook op basis van de vertrouwensrelaties die er zijn tussen de systemen.
Er wordt gezocht naar een oplossing om de vertrouwde rootcertificaten centraal te beheren, met de mogelijkheid om deze te beperken tot enkel de noodzakelijke vertrouwensrelaties, bijvoorbeeld in het kader van netwerksegmentatie.

\section{\IfLanguageName{dutch}{Onderzoeksvraag}{Research question}}%
\label{sec:onderzoeksvraag}

Aan de hand van welke combinatie van tools en configuraties kan een trust management-systeem worden opgezet binnen een bedrijf met een heterogeen gesegmenteerd netwerk waarbij de inhoud van de truststores van endpoints centraal kan worden beheerd?

Deelvragen:
\begin{itemize}
    \item Wat zijn de belangrijkste concepten en technologieën achter PKI en truststorebeheer? 
    \item Wat zijn de verschillende truststores die worden gebruikt door de meest voorkomende besturingssystemen? Hoe kunnen deze worden beheerd?
    \item Welke tools en technieken kunnen worden gebruikt voor truststorebeheer, en wat zijn hun beperkingen? 
    \item Hoe kan de inhoud van de truststores van endpoints centraal worden beheerd?
    \item Hoe kan men de vertrouwde certificaten anders beheren voor verschillende netwerksegmenten?
    \item Welk netwerkplan zal worden gebruikt om de proof-of-concept virtuele omgeving op te zetten?
    \item Hoe worden de gevonden tools en technieken geïmplementeerd in de proof-of-concept virtuele omgeving?
    \item Wat is de effectiviteit van de opgezette proof-of-concept virtuele omgeving?
    \item Welke aanbevelingen kunnen worden gegeven aan bedrijven die een trust management-systeem willen implementeren?
\end{itemize}

\section{\IfLanguageName{dutch}{Onderzoeksdoelstelling}{Research objective}}%
\label{sec:onderzoeksdoelstelling}

Het doel binnen dit onderzoek is om een proof-of-concept virtuele omgeving op te zetten die systemen met de meest commercieel gebruikte besturingssystemen en netweksegmentatie, waarbij de truststores van de systemen centraal kan worden beheerd en hun inhoud afhankelijk is van het netwerksegment waar ze zich in bevinden.
De proof-of-concept omgeving zal als inspiratie dienen voor bedrijven die een trust management systeem willen implementeren, de nadruk binnen dit onderzoek wordt gelegd op de functionaliteit van de oplossing, er zal dus niet te diep gekeken worden naar de beveiliging van de oplossing.
Verdere aanbevelingen voor een implementatie in een productieomgeving zullen ook worden gegeven aan bedrijven.

\section{\IfLanguageName{dutch}{Opzet van deze bachelorproef}{Structure of this bachelor thesis}}%
\label{sec:opzet-bachelorproef}

% Het is gebruikelijk aan het einde van de inleiding een overzicht te
% geven van de opbouw van de rest van de tekst. Deze sectie bevat al een aanzet
% die je kan aanvullen/aanpassen in functie van je eigen tekst.

De rest van deze bachelorproef is als volgt opgebouwd:

In Hoofdstuk~\ref{ch:stand-van-zaken} wordt een overzicht gegeven van de stand van zaken binnen het onderzoeksdomein, op basis van een literatuurstudie.

In Hoofdstuk~\ref{ch:methodologie} wordt de methodologie toegelicht en worden de gebruikte onderzoekstechnieken besproken om een antwoord te kunnen formuleren op de onderzoeksvragen.

% TODO: Vul hier aan voor je eigen hoofstukken, één of twee zinnen per hoofdstuk

In Hoofdstuk~\ref{ch:proof-of-concept} wordt een virtuele omgeving opgesteld die aan de hand van de gevonden tools en technieken uit de literatuurstudie een aantal werkende trust management systemen bevat. Aanbevelingen worden meegegeven aan bedrijven voor de implementatie van deze systemen in hun infrastructuur.

In Hoofdstuk~\ref{ch:conclusie}, tenslotte, wordt de conclusie gegeven en een antwoord geformuleerd op de onderzoeksvragen. Daarbij wordt ook een aanzet gegeven voor toekomstig onderzoek binnen dit domein.