%%=============================================================================
%% Inleiding
%%=============================================================================

\chapter{\IfLanguageName{dutch}{Inleiding}{Introduction}}%
\label{ch:inleiding}

In moderne bedrijfsomgevingen worden netwerken steeds complexer en diverser. Organisaties bevatten vaak heterogene netwerken, dat wilt zeggen dat binnen het netwerken verschillende soorten apparaten aanwezig zijn die niet allemaal dezelfde besturingssystemen en software gebruiken. 
Vaak zijn deze netwerken ook opgedeeld in verschillende segmenten om de impact van een eventuele aanval te beperken.
Hoewel segmentatie helpt om bedreigingen te beperken, brengt het ook uitdagingen met zich mee op het gebied van toegangsbeheer en welke bronnen men kan vertrouwen en bereiken binnen de verschillende segmenten. 
Een specifiek probleem binnen deze context is dat niet elk netwerksegment dezelfde root certificates wil vertrouwen, wat een aanzienlijk beveiligingsrisico met zich meebrengt.

Trust management biedt een mogelijke oplossing door dynamische en contextbewuste toegang te faciliteren, waarbij beslissingen worden genomen op basis van factoren zoals gebruikersidentiteit, gedragsanalyse en beleidsregels. Een belangrijke uitdaging is echter hoe de trust stores van endpoints in een netwerk centraal kunnen worden beheerd om dit probleem effectief aan te pakken. De implementatie van een effectief trust management-systeem in een heterogeen, gesegmenteerd netwerk vereist een grondige analyse van de risico's, beleidsmodellen en technologische vereisten.

\section{\IfLanguageName{dutch}{Probleemstelling}{Problem Statement}}%
\label{sec:probleemstelling}

Vele bedrijven met grotere diverse netwerken zoals heterogene netwerken en gesegmenteerde netwerken hebben vandaag de dag nogsteeds moeite met het centraal beheren van de (root) certificaten die worden vertrouwd door hun endpoints. 
Dit komt door de verspreide trust stores die gevormt worden door de verschillende gebruikte applicaties en besturingssystemen binnen het netwerk, daarnaast is het ook belangrijk om het aantal vertrouwde certificaten te beperken om de veiligheid van het netwerk te garanderen.
Vele bedrijven hebben een netwerksegmentatie toegepast afhankelijk van hun bedrijfsrollen en -vereisten, maar dit brengt ook uitdagingen met zich mee op vlak van welke segmenten welke certificates vertrouwen en hoe dit kan worden afgedwongen.

\section{\IfLanguageName{dutch}{Onderzoeksvraag}{Research question}}%
\label{sec:onderzoeksvraag}

Aan de hand van welke combinatie van tools en configuraties kan een trust management-systeem worden opgezet binnen een bedrijf met een heterogeen gesegmenteerd netwerk waarbij de inhoud van de trust stores van endpoints makkelijk en snel centraal kan worden beheerd?

Deelvragen:
\begin{itemize}
    \item Wat zijn de belangrijkste concepten en technologieën achter PKI en truststorebeheer? 
    \item Wat zijn de verschillende truststores die worden gebruikt door de meest voorkomende besturingssystemen en applicaties? Hoe kunnen deze worden beheerd?
    \item Welke tools en technieken worden momenteel gebruikt voor truststorebeheer, en wat zijn hun beperkingen? 
    \item Hoe kan de inhoud van de trust stores van endpoints centraal worden beheerd?
    \item Hoe kan men de vertrouwde certificaten anders beheren binnen verschillende netwerksegmenten?
    \item Welk architectuurplan zal worden gebruikt om de proof-of-concept virtuele omgeving op te zetten?
    \item Hoe worden de gevonden tools en technieken geïmplementeerd in de proof-of-concept virtuele omgeving?
    \item Wat is de effectiviteit van de opgezette proof-of-concept virtuele omgeving?
    \item Welke aspecten zijn niet opgenomen in de proof-of-concept virtuele omgeving die in realiteit voor problemen kunnen zorgen?
    \item Welke aanbevelingen kunnen worden gegeven aan bedrijven die een trust management-systeem willen implementeren?
\end{itemize}

\section{\IfLanguageName{dutch}{Onderzoeksdoelstelling}{Research objective}}%
\label{sec:onderzoeksdoelstelling}

Het doel binnen dit onderzoek is om een proof-of-concept virtuele omgeving op te zetten die systemen met veel voorkomende besturingssystemen en applicaties bevat, alsook netwerksegmentatie waarbij de trust stores van de endpoints centraal kunnen worden beheerd.

\section{\IfLanguageName{dutch}{Opzet van deze bachelorproef}{Structure of this bachelor thesis}}%
\label{sec:opzet-bachelorproef}

% Het is gebruikelijk aan het einde van de inleiding een overzicht te
% geven van de opbouw van de rest van de tekst. Deze sectie bevat al een aanzet
% die je kan aanvullen/aanpassen in functie van je eigen tekst.

De rest van deze bachelorproef is als volgt opgebouwd:

In Hoofdstuk~\ref{ch:stand-van-zaken} wordt een overzicht gegeven van de stand van zaken binnen het onderzoeksdomein, op basis van een literatuurstudie.

In Hoofdstuk~\ref{ch:methodologie} wordt de methodologie toegelicht en worden de gebruikte onderzoekstechnieken besproken om een antwoord te kunnen formuleren op de onderzoeksvragen.

% TODO: Vul hier aan voor je eigen hoofstukken, één of twee zinnen per hoofdstuk

In Hoofdstuk~\ref{ch:conclusie}, tenslotte, wordt de conclusie gegeven en een antwoord geformuleerd op de onderzoeksvragen. Daarbij wordt ook een aanzet gegeven voor toekomstig onderzoek binnen dit domein.